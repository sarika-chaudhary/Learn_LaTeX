\documentclass[12pt, a4paper]{article}

\usepackage[margin=1in]{geometry}
\usepackage{amsfonts,amssymb, amsmath}
\usepackage{tikz,pgfplots}
\usepackage{graphicx}
\usepackage{float}


\def\eq1{y=\dfrac{x}{3x^2+x+1}}
\newcommand{\set}[1]{\setlength\itemsep{#1em}}
\newcommand\calculator{\tikz{
		\node (c) [inner sep=0pt, draw, fill=black, anchor=south west]{\phantom{N}};
		\begin{scope}[x=(c.south east),y=(c.north west)]    \fill[white] (.1,.7) rectangle (.9,.9);    
		\foreach \x in {.1, .33, .55, .79}{    
		\foreach \y in {.1, .24, .38, .53}{    
		\fill[white] (\x,\y) rectangle +(.11,.07);}} 
		\end{scope} }}
		\def\calcicon#1{\noindent#1 \calculator\ }




\begin{document}



\textbf{WHAT IS HISTORY?}
\begin{enumerate}
\set{1.2}
\item\calculator\ Let us calculate 2+2.
\item This is the symbol for the set of all real number: $\mathbb{R}$.
\item This is the symbol for the set of integers: $\mathbb{Z}$.
\item This is the symbol for the set of all rational number $\mathbb{Q}$
\item What is the need to learn history?
\item Are there one history or several histories?
\item Who has the power to write history?
\end{enumerate}
\vspace{1cm}
we can also use this instead of margin.
[top=1in,botton=1in,right=0.5in,left=0.5in]

\vspace{1cm}
You can make something comment by using percent symbol. 
% This won't print. 
But this would.

\vspace{1cm}
micros 
\begin{enumerate}
\item Let's examination equation $\eq1$.
\end{enumerate}

\vspace{1cm}
inserting graphics 
\begin{enumerate}
\item Let's put the image here.\\ 

\includegraphics[scale=0.2]{sarika}\\


\vspace{1cm}
\begin{center}
\includegraphics[width=0.2\textwidth]{sarika}
\end{center}
\end{enumerate}

\vspace{1cm}

\begin{figure}[H]
\centering
\includegraphics[width=0.2\textwidth]{sarika}
\caption{This is me}

\end{figure}

\end{document}